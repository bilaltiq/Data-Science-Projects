% Options for packages loaded elsewhere
\PassOptionsToPackage{unicode}{hyperref}
\PassOptionsToPackage{hyphens}{url}
\PassOptionsToPackage{dvipsnames,svgnames,x11names}{xcolor}
%
\documentclass[
  letterpaper,
  DIV=11,
  numbers=noendperiod]{scrartcl}

\usepackage{amsmath,amssymb}
\usepackage{iftex}
\ifPDFTeX
  \usepackage[T1]{fontenc}
  \usepackage[utf8]{inputenc}
  \usepackage{textcomp} % provide euro and other symbols
\else % if luatex or xetex
  \usepackage{unicode-math}
  \defaultfontfeatures{Scale=MatchLowercase}
  \defaultfontfeatures[\rmfamily]{Ligatures=TeX,Scale=1}
\fi
\usepackage{lmodern}
\ifPDFTeX\else  
    % xetex/luatex font selection
\fi
% Use upquote if available, for straight quotes in verbatim environments
\IfFileExists{upquote.sty}{\usepackage{upquote}}{}
\IfFileExists{microtype.sty}{% use microtype if available
  \usepackage[]{microtype}
  \UseMicrotypeSet[protrusion]{basicmath} % disable protrusion for tt fonts
}{}
\makeatletter
\@ifundefined{KOMAClassName}{% if non-KOMA class
  \IfFileExists{parskip.sty}{%
    \usepackage{parskip}
  }{% else
    \setlength{\parindent}{0pt}
    \setlength{\parskip}{6pt plus 2pt minus 1pt}}
}{% if KOMA class
  \KOMAoptions{parskip=half}}
\makeatother
\usepackage{xcolor}
\setlength{\emergencystretch}{3em} % prevent overfull lines
\setcounter{secnumdepth}{-\maxdimen} % remove section numbering
% Make \paragraph and \subparagraph free-standing
\ifx\paragraph\undefined\else
  \let\oldparagraph\paragraph
  \renewcommand{\paragraph}[1]{\oldparagraph{#1}\mbox{}}
\fi
\ifx\subparagraph\undefined\else
  \let\oldsubparagraph\subparagraph
  \renewcommand{\subparagraph}[1]{\oldsubparagraph{#1}\mbox{}}
\fi

\usepackage{color}
\usepackage{fancyvrb}
\newcommand{\VerbBar}{|}
\newcommand{\VERB}{\Verb[commandchars=\\\{\}]}
\DefineVerbatimEnvironment{Highlighting}{Verbatim}{commandchars=\\\{\}}
% Add ',fontsize=\small' for more characters per line
\usepackage{framed}
\definecolor{shadecolor}{RGB}{241,243,245}
\newenvironment{Shaded}{\begin{snugshade}}{\end{snugshade}}
\newcommand{\AlertTok}[1]{\textcolor[rgb]{0.68,0.00,0.00}{#1}}
\newcommand{\AnnotationTok}[1]{\textcolor[rgb]{0.37,0.37,0.37}{#1}}
\newcommand{\AttributeTok}[1]{\textcolor[rgb]{0.40,0.45,0.13}{#1}}
\newcommand{\BaseNTok}[1]{\textcolor[rgb]{0.68,0.00,0.00}{#1}}
\newcommand{\BuiltInTok}[1]{\textcolor[rgb]{0.00,0.23,0.31}{#1}}
\newcommand{\CharTok}[1]{\textcolor[rgb]{0.13,0.47,0.30}{#1}}
\newcommand{\CommentTok}[1]{\textcolor[rgb]{0.37,0.37,0.37}{#1}}
\newcommand{\CommentVarTok}[1]{\textcolor[rgb]{0.37,0.37,0.37}{\textit{#1}}}
\newcommand{\ConstantTok}[1]{\textcolor[rgb]{0.56,0.35,0.01}{#1}}
\newcommand{\ControlFlowTok}[1]{\textcolor[rgb]{0.00,0.23,0.31}{#1}}
\newcommand{\DataTypeTok}[1]{\textcolor[rgb]{0.68,0.00,0.00}{#1}}
\newcommand{\DecValTok}[1]{\textcolor[rgb]{0.68,0.00,0.00}{#1}}
\newcommand{\DocumentationTok}[1]{\textcolor[rgb]{0.37,0.37,0.37}{\textit{#1}}}
\newcommand{\ErrorTok}[1]{\textcolor[rgb]{0.68,0.00,0.00}{#1}}
\newcommand{\ExtensionTok}[1]{\textcolor[rgb]{0.00,0.23,0.31}{#1}}
\newcommand{\FloatTok}[1]{\textcolor[rgb]{0.68,0.00,0.00}{#1}}
\newcommand{\FunctionTok}[1]{\textcolor[rgb]{0.28,0.35,0.67}{#1}}
\newcommand{\ImportTok}[1]{\textcolor[rgb]{0.00,0.46,0.62}{#1}}
\newcommand{\InformationTok}[1]{\textcolor[rgb]{0.37,0.37,0.37}{#1}}
\newcommand{\KeywordTok}[1]{\textcolor[rgb]{0.00,0.23,0.31}{#1}}
\newcommand{\NormalTok}[1]{\textcolor[rgb]{0.00,0.23,0.31}{#1}}
\newcommand{\OperatorTok}[1]{\textcolor[rgb]{0.37,0.37,0.37}{#1}}
\newcommand{\OtherTok}[1]{\textcolor[rgb]{0.00,0.23,0.31}{#1}}
\newcommand{\PreprocessorTok}[1]{\textcolor[rgb]{0.68,0.00,0.00}{#1}}
\newcommand{\RegionMarkerTok}[1]{\textcolor[rgb]{0.00,0.23,0.31}{#1}}
\newcommand{\SpecialCharTok}[1]{\textcolor[rgb]{0.37,0.37,0.37}{#1}}
\newcommand{\SpecialStringTok}[1]{\textcolor[rgb]{0.13,0.47,0.30}{#1}}
\newcommand{\StringTok}[1]{\textcolor[rgb]{0.13,0.47,0.30}{#1}}
\newcommand{\VariableTok}[1]{\textcolor[rgb]{0.07,0.07,0.07}{#1}}
\newcommand{\VerbatimStringTok}[1]{\textcolor[rgb]{0.13,0.47,0.30}{#1}}
\newcommand{\WarningTok}[1]{\textcolor[rgb]{0.37,0.37,0.37}{\textit{#1}}}

\providecommand{\tightlist}{%
  \setlength{\itemsep}{0pt}\setlength{\parskip}{0pt}}\usepackage{longtable,booktabs,array}
\usepackage{calc} % for calculating minipage widths
% Correct order of tables after \paragraph or \subparagraph
\usepackage{etoolbox}
\makeatletter
\patchcmd\longtable{\par}{\if@noskipsec\mbox{}\fi\par}{}{}
\makeatother
% Allow footnotes in longtable head/foot
\IfFileExists{footnotehyper.sty}{\usepackage{footnotehyper}}{\usepackage{footnote}}
\makesavenoteenv{longtable}
\usepackage{graphicx}
\makeatletter
\def\maxwidth{\ifdim\Gin@nat@width>\linewidth\linewidth\else\Gin@nat@width\fi}
\def\maxheight{\ifdim\Gin@nat@height>\textheight\textheight\else\Gin@nat@height\fi}
\makeatother
% Scale images if necessary, so that they will not overflow the page
% margins by default, and it is still possible to overwrite the defaults
% using explicit options in \includegraphics[width, height, ...]{}
\setkeys{Gin}{width=\maxwidth,height=\maxheight,keepaspectratio}
% Set default figure placement to htbp
\makeatletter
\def\fps@figure{htbp}
\makeatother

\usepackage{booktabs}
\usepackage{longtable}
\usepackage{array}
\usepackage{multirow}
\usepackage{wrapfig}
\usepackage{float}
\usepackage{colortbl}
\usepackage{pdflscape}
\usepackage{tabu}
\usepackage{threeparttable}
\usepackage{threeparttablex}
\usepackage[normalem]{ulem}
\usepackage{makecell}
\usepackage{xcolor}
\KOMAoption{captions}{tableheading}
\makeatletter
\makeatother
\makeatletter
\makeatother
\makeatletter
\@ifpackageloaded{caption}{}{\usepackage{caption}}
\AtBeginDocument{%
\ifdefined\contentsname
  \renewcommand*\contentsname{Table of contents}
\else
  \newcommand\contentsname{Table of contents}
\fi
\ifdefined\listfigurename
  \renewcommand*\listfigurename{List of Figures}
\else
  \newcommand\listfigurename{List of Figures}
\fi
\ifdefined\listtablename
  \renewcommand*\listtablename{List of Tables}
\else
  \newcommand\listtablename{List of Tables}
\fi
\ifdefined\figurename
  \renewcommand*\figurename{Figure}
\else
  \newcommand\figurename{Figure}
\fi
\ifdefined\tablename
  \renewcommand*\tablename{Table}
\else
  \newcommand\tablename{Table}
\fi
}
\@ifpackageloaded{float}{}{\usepackage{float}}
\floatstyle{ruled}
\@ifundefined{c@chapter}{\newfloat{codelisting}{h}{lop}}{\newfloat{codelisting}{h}{lop}[chapter]}
\floatname{codelisting}{Listing}
\newcommand*\listoflistings{\listof{codelisting}{List of Listings}}
\makeatother
\makeatletter
\@ifpackageloaded{caption}{}{\usepackage{caption}}
\@ifpackageloaded{subcaption}{}{\usepackage{subcaption}}
\makeatother
\makeatletter
\@ifpackageloaded{tcolorbox}{}{\usepackage[skins,breakable]{tcolorbox}}
\makeatother
\makeatletter
\@ifundefined{shadecolor}{\definecolor{shadecolor}{rgb}{.97, .97, .97}}
\makeatother
\makeatletter
\makeatother
\makeatletter
\makeatother
\ifLuaTeX
  \usepackage{selnolig}  % disable illegal ligatures
\fi
\IfFileExists{bookmark.sty}{\usepackage{bookmark}}{\usepackage{hyperref}}
\IfFileExists{xurl.sty}{\usepackage{xurl}}{} % add URL line breaks if available
\urlstyle{same} % disable monospaced font for URLs
\hypersetup{
  pdftitle={Lab2\_DatasauRus},
  pdfauthor={Bilal Tariq},
  colorlinks=true,
  linkcolor={blue},
  filecolor={Maroon},
  citecolor={Blue},
  urlcolor={Blue},
  pdfcreator={LaTeX via pandoc}}

\title{Lab2\_DatasauRus}
\author{Bilal Tariq}
\date{2024-01-31}

\begin{document}
\maketitle
\ifdefined\Shaded\renewenvironment{Shaded}{\begin{tcolorbox}[enhanced, boxrule=0pt, borderline west={3pt}{0pt}{shadecolor}, sharp corners, frame hidden, interior hidden, breakable]}{\end{tcolorbox}}\fi

\begin{Shaded}
\begin{Highlighting}[]
\FunctionTok{library}\NormalTok{(tidyverse)}
\FunctionTok{library}\NormalTok{(kableExtra)}
\FunctionTok{library}\NormalTok{(datasauRus)}

\CommentTok{\# set black \& white default plot theme}
\FunctionTok{theme\_set}\NormalTok{(}\FunctionTok{theme\_classic}\NormalTok{()) }

\CommentTok{\# improve digit and NA display }
\FunctionTok{options}\NormalTok{(}\AttributeTok{scipen =} \DecValTok{1}\NormalTok{, }\AttributeTok{knitr.kable.NA =} \StringTok{\textquotesingle{}\textquotesingle{}}\NormalTok{)}
\end{Highlighting}
\end{Shaded}

\hypertarget{lab-purpose}{%
\section{Lab Purpose}\label{lab-purpose}}

This lab is a modified version of a lab developed by Mine
Cetinkaya-Rundel. You'll be working in random breakout groups to work
through the lab.

\emph{Introduce yourselves to each other as you get started - preferred
names, at least one academic interest, and at least one extracurricular
activity.}

The main goal of this lab is to introduce you to working with R and
RStudio in conjunction with git and GitHub, all of which we will be
using throughout the course. You will then submit the completed lab to
Gradescope (the compiled .pdf of it) to make sure everyone is able to
use Gradescope appropriately. Further directions are contained in the
Practice0 assignment and you should check them out.

We will work primarily with two R packages: \texttt{datasauRus} which
contains the datasets for the lab, and \texttt{tidyverse} which is a
collection of packages for doing data analysis in a ``tidy'' way. Both
packages are ready to be loaded in the \texttt{setup} code chunk.
\texttt{tidyverse} loads \texttt{dplyr} which many of you may have seen
used for data wrangling in other stats courses. It also loads
\texttt{ggplot2} which is the package we'll be using to make our
visualizations (more next week).

\hypertarget{step-0-get-the-lab-.qmd-file-from-the-course-git-repo-by-pulling-the-repo.-copy-the-file-over-to-your-personal-repo-for-class.}{%
\subsection{Step 0: Get the lab .Qmd file from the course git repo by
pulling the repo. Copy the file over to your personal repo for
class.}\label{step-0-get-the-lab-.qmd-file-from-the-course-git-repo-by-pulling-the-repo.-copy-the-file-over-to-your-personal-repo-for-class.}}

Assuming you have Prep0 completed (you should!), and you accepted your
invite to the class GitHub organization, you should have access to the
course git repo. Pulling updates your local copy of the repo, making
sure you have the most recent copy of the lab that I've posted. Then,
you can copy the file over to your repo to work on it. More detailed
directions for how to do the copying are in the GitHub classroom guide.
Once this is done, we are ready to get going.

\hypertarget{step-1-verify-you-are-working-on-the-lab-saved-in-your-repo-and-that-you-are-working-in-your-project-repo.}{%
\subsection{\texorpdfstring{Step 1: Verify you are working on the lab
saved in \emph{your} repo and that you are working in \emph{your}
project
repo.}{Step 1: Verify you are working on the lab saved in your repo and that you are working in your project repo.}}\label{step-1-verify-you-are-working-on-the-lab-saved-in-your-repo-and-that-you-are-working-in-your-project-repo.}}

Look in the top right of RStudio to verify the project name matches
\emph{your} repo. (If you decided to use the GitHub Desktop app, you
won't need this part; instead check in your file explorer that you're
working in the right place.)

Knit the Qmd/Rmd file (\texttt{Cmd\ +\ Shift\ +\ K} on Mac or
\texttt{Ctrl\ +\ Shift\ +\ K} on PC or hit the \texttt{Knit} button in
the toolbar above) and verify the pdf (or other relevant files) is
created in \emph{your} repo and not some other location.

(From here on, if you see Rmd, it's because it could be Qmd or Rmd -
it's whatever file you're coding in, etc.)

\hypertarget{step-2-enter-your-name-in-the-yaml-header-where-indicated.}{%
\subsection{Step 2: Enter your name in the YAML header where
indicated.}\label{step-2-enter-your-name-in-the-yaml-header-where-indicated.}}

``YAML'' rhymes with ``camel'' and stands for ``Yet Another Markup
Language''. YAML headers are very sensitive - even a space or comma out
of place can cause them to break. Be careful if you delete the
parentheses there - they must be put back exactly.

\hypertarget{step-3-knit-view-diff-and-commit}{%
\subsection{Step 3: Knit, view diff, and
commit!}\label{step-3-knit-view-diff-and-commit}}

Entering your name made a change to the file that we will be able to
track with Git.

Knit the Rmd file, then go to your Github Desktop app (or the Git pane
if you set it up in RStudio).

Click on the document in the app to see the changes you made. You can
uncheck the .pdf at the same time - we only want the .Qmd/.Rmd at the
moment.

If you're happy with the changes, write ``Update author name'' in the
\textbf{Commit message} box, and hit \textbf{Commit}. The box is in the
lower left corner of the app.

This saves your changes locally including the message. The changes
haven't been sent back to GitHub yet though.

\emph{How often should you commit??} You do not have to commit after
every change; this would get quite cumbersome. You should consider
committing states that are \emph{meaningful to you} for inspection,
comparison, or restoration. For example, perhaps you commit after you've
completed part of a problem, or after you got a graphic working, or a
really complicated wrangling command down.

\hypertarget{step-4-push-your-changes}{%
\subsection{Step 4: Push your changes}\label{step-4-push-your-changes}}

After committing, go ahead and \textbf{Push} your commit(s) back onto
GitHub using the buttons. In the app, after you've made a commit, the
Fetch/Pull option turns into a Push option. Click that to push.

You can think of commits as snapshots of your work over time, and
pushing will sync your work with GitHub so you (or a collaborator) can
pick up where you left off but on another device. You don't have to push
after every commit, but should push before you stop working on anything.
You want your changes to be available to the next person working (your
future self or your collaborators).

\hypertarget{step-5-edit-commit-and-push-until-done}{%
\subsection{Step 5: Edit, Commit, and Push until
done!}\label{step-5-edit-commit-and-push-until-done}}

Work through the rest of the lab. When you are done with the assignment,
save the pdf as
``\emph{YourFirstInitialYourLastName}\_\emph{thisfilename}.pdf'' before
committing and pushing (this is generally good practice but also helps
me in those times where I need to download all student homework files).
For example, I would save this as AWagaman\_Lab2.pdf. At times it is
also useful to add the date at the end. Final pdfs should be relabeled
this way to set them apart as your final submission.

You MUST push a final version of each lab .Qmd.Rmd and .pdf to your repo
for our course assignments. This counts as part of your participation in
class, and for this assignment, part of Practice0 is making sure I can
find these in your repo. Note that you don't have to push the .pdf until
you've finalized it, but it doesn't hurt to do it earlier either
(assuming you keep the repo organized, anyway). Just be sure you get the
rename in on the final version so that's clear.

\hypertarget{gradescope-upload}{%
\subsection{Gradescope Upload}\label{gradescope-upload}}

Our Practice0 assignment is designed to ensure you've successfully
submitted an assignment to Gradescope before the first real assignment
is due. No feedback will be provided on this assignment. Here's how that
will work when you are finished with the lab and have your final .pdf.

For each question/part, allocate all pages associated with the specific
question/part. Do your best with this. If your work for a question runs
onto a page that you did not select, you may not get credit for the
work. If you do n?ot allocate \emph{any} pages when you upload your pdf,
you may get a zero for the assignment.

You can resubmit your work as many times as you want before the
deadline, so you should absolutely not wait until the last minute to
submit some version of your work. You should be compiling as you go,
reducing the chance of last minute compiling issues. Your git repo will
show when you've made commits, too. For this activity, I've provided the
commit messages so they should show. If you do run into an issue, submit
whatever you have completed to receive partial credit.

\hypertarget{come-together---make-sure-everyone-is-ready-before-moving-on}{%
\subsection{Come together - Make sure everyone is ready before moving
on}\label{come-together---make-sure-everyone-is-ready-before-moving-on}}

\newpage

\hypertarget{working-with-some-data-and-practicing-workflow}{%
\section{Working with some data and practicing
workflow}\label{working-with-some-data-and-practicing-workflow}}

The data we will be working with is called \emph{datasaurus\_dozen} and
it's in the \texttt{datasauRus} package.

When loaded, this data appears to be a single dataset. However, this
single dataset contains 13 datasets, designed to show us why data
visualisation is important and how summary statistics alone can be
misleading. The different datasets are identified by the
\texttt{dataset} variable.

To find out more about the dataset, type the following in your Console:
\texttt{?datasaurus\_dozen}. A question mark before the name of an
object will always bring up its help file. This command can be run in
the Console or a code chunk, but remove it or comment it out from the
code chunk before you compile.

\hypertarget{understanding-the-data}{%
\section{1 - Understanding the data}\label{understanding-the-data}}

It is tempting to jump into visualization and analysis (like we did a
little on the first day), but it is critical as a statistician that we
first understand the context and structure of the data.

\begin{quote}
part a - Based on the help file, how many rows and how many columns are
in the data frame?
\end{quote}

Solution: There are 1846 rows with 3 columns.

\begin{quote}
part b - Based on the help file, what variables are included in the data
frame?
\end{quote}

Solution: There are 3 variables: the dataset, x and y values.

Markdown allows you to do a lot in terms of formatting that you may not
have seen before. For example, you can enter bulleted lists by using the
following pattern:

\begin{itemize}
\tightlist
\item
  first item
\item
  second item
\item
  third item
\end{itemize}

For more on Markdown formatting, you can check out the RMarkdown
formatting \texttt{cheatsheet} in our Resources folder or find out more
\href{https://www.rstudio.com/resources/cheatsheets/}{here} online.
You'll see lots of cheatsheets available for reference.

\begin{quote}
part c - Use Markdown formatting to provide a bulleted list of the
variables.
\end{quote}

Solution:

\begin{quote}
part d - How many observations are in each dataset within this larger
data frame? Let's make a \emph{frequency table} of the of the
\texttt{dataset} variable to find out.
\end{quote}

Note: The \texttt{kable()} function (from \texttt{knitr} package, with
more functionality available from the \texttt{kableExtra} package) makes
nicer looking tables when you knit. For quick ``professional quality''
pdf tables, add the argument \texttt{booktabs\ =\ TRUE}.

Try knitting the document without piping the frequency table to
\texttt{kable()}, with the pipe to \texttt{kable()}, and finally with
the pipe to \texttt{kable(booktabs\ =\ TRUE)} to see the differences.

\begin{Shaded}
\begin{Highlighting}[]
\NormalTok{datasaurus\_dozen }\SpecialCharTok{\%\textgreater{}\%} 
  \FunctionTok{count}\NormalTok{(dataset) }\SpecialCharTok{\%\textgreater{}\%} 
  \FunctionTok{kable}\NormalTok{(}\AttributeTok{booktabs =} \ConstantTok{TRUE}\NormalTok{) }
\end{Highlighting}
\end{Shaded}

\begin{longtable}[]{@{}lr@{}}
\toprule\noalign{}
dataset & n \\
\midrule\noalign{}
\endhead
\bottomrule\noalign{}
\endlastfoot
away & 142 \\
bullseye & 142 \\
circle & 142 \\
dino & 142 \\
dots & 142 \\
h\_lines & 142 \\
high\_lines & 142 \\
slant\_down & 142 \\
slant\_up & 142 \\
star & 142 \\
v\_lines & 142 \\
wide\_lines & 142 \\
x\_shape & 142 \\
\end{longtable}

Note: By default, the table is left-justified. You can pipe to an
additional function called \texttt{kable\_styling()} which allows
further customization of pdf tables. By default, this function will
center the table. It may also do a weird LaTeX thing where the table
ends up somewhere else in the document other than where you want it. To
prevent that from happening, we can add the argument
\texttt{latex\_options\ =\ "hold\_position"} (or
\texttt{"HOLD\_position"} if you're \emph{really} serious).

LaTeX is responsible for setting up our document formatting.
Kniting/compiling actually creates a .tex document first that is
converted into a .pdf. Learning some LaTeX can be very useful for trying
to deal with issues as well as getting equations and symbols in the
document in the format you want.

Solution: (How many observations are in each data set? Write a
sentence!)

\newpage

\hypertarget{data-visualization-and-summary}{%
\section{2 - Data visualization and
summary}\label{data-visualization-and-summary}}

\begin{quote}
part a - A correlation
\end{quote}

Calculate the correlation coefficient, \(r\), between \texttt{x} and
\texttt{y} for the \texttt{dino} dataset. Below is the code you will
need to complete this exercise. Basically, the answer is already given,
but you need to include relevant bits in your Rmd document and
successfully knit it and view the results.

\begin{Shaded}
\begin{Highlighting}[]
\CommentTok{\# Start with \textasciigrave{}datasaurus\_dozen\textasciigrave{}}
\CommentTok{\# Filter for observations where \textasciigrave{}dataset == "dino"\textasciigrave{}}
\CommentTok{\# Store the resulting filtered data frame as a new data frame called \textasciigrave{}dino\_data\textasciigrave{}}
\NormalTok{dino\_data }\OtherTok{\textless{}{-}}\NormalTok{ datasaurus\_dozen }\SpecialCharTok{\%\textgreater{}\%}
  \FunctionTok{filter}\NormalTok{(dataset }\SpecialCharTok{==} \StringTok{"dino"}\NormalTok{)}

\CommentTok{\# Compute correlation between \textasciigrave{}x\textasciigrave{} and \textasciigrave{}y\textasciigrave{} for \textasciigrave{}dino\textasciigrave{} dataset with label \textasciigrave{}r\textasciigrave{}}
\NormalTok{dino\_data }\SpecialCharTok{\%\textgreater{}\%}
  \FunctionTok{summarize}\NormalTok{(}\AttributeTok{r =} \FunctionTok{cor}\NormalTok{(x, y))}
\end{Highlighting}
\end{Shaded}

\begin{verbatim}
# A tibble: 1 x 1
        r
    <dbl>
1 -0.0645
\end{verbatim}

What does this correlation coefficient tell us about the relationship
between \texttt{x} and \texttt{y} in the \texttt{dino} dataset?

Solution:

\begin{quote}
part b - A plot
\end{quote}

Oops. In intro stats, we learned correlation is an appropriate measure
to use to assess linear relationships. But we didn't plot the
relationship here to check it out first. Let's do that.

We want to plot \texttt{y} vs.~\texttt{x} for the \texttt{dino} dataset
using the \texttt{ggplot()} function. Its first argument is the data
you're visualizing. Next we define the \texttt{aes}thetic mappings. In
other words, the columns of the data that get mapped to certain
aesthetic features of the plot, e.g.~the \(x\) axis will represent the
variable called \texttt{x} and the \(y\) axis will represent the
variable called \texttt{y}. Then, we add another layer to this plot
where we define which \texttt{geom}etric shapes we want to use to
represent each observation in the data. In this case we want these to be
points, hence \texttt{geom\_point}.

You will learn about the philosophy of building data visualizations in
detail next week. For now, follow along with the code that is provided.

\begin{Shaded}
\begin{Highlighting}[]
\FunctionTok{ggplot}\NormalTok{(}\AttributeTok{data =}\NormalTok{ dino\_data, }\AttributeTok{mapping =} \FunctionTok{aes}\NormalTok{(}\AttributeTok{x =}\NormalTok{ x, }\AttributeTok{y =}\NormalTok{ y)) }\SpecialCharTok{+}
  \FunctionTok{geom\_point}\NormalTok{()}
\end{Highlighting}
\end{Shaded}

What do you notice now about the relationship between x and y? Is using
correlation as a summary measure appropriate?

Solution:

\begin{quote}
part c - Another correlation
\end{quote}

Now calculate the correlation coefficient between \texttt{x} and
\texttt{y} for the \texttt{star} dataset. You can (and should) reuse
code we introduced above, after pasting it below, just replace the
dataset name with the desired dataset. How does this value compare to
the correlation coefficient from the \texttt{dino} dataset?

Solution:

\begin{quote}
part d - Another plot
\end{quote}

Plot \texttt{y} versus \texttt{x} for the \texttt{star} dataset. Does
the plot look the same as the plot of the \texttt{dino} data? Is
correlation an appropriate summary measure?

Solution:

\begin{quote}
part e - A final correlation
\end{quote}

Now calculate the correlation coefficient between \texttt{x} and
\texttt{y} for the \texttt{circle} dataset. You can (and should) reuse
code we introduced above, after pasting it below, just replace the
dataset name with the desired dataset. How does this value compare to
the correlation coefficients from the other two datasets?

Solution:

\begin{quote}
part f - A final plot
\end{quote}

Plot \texttt{y} versus \texttt{x} for the \texttt{circle} dataset. Does
the plot look the same as either plot from the other two datasets?

Solution:

\newpage

\hypertarget{making-things-more-efficient}{%
\section{3 - Making things more
efficient}\label{making-things-more-efficient}}

The previous problem had a lot of repetition. To make things more
efficient, we can plot all the datasets at once using \emph{facets}, and
we can compute all the correlations at once using the
\texttt{group\_by()} function. The code is provided below.

\begin{quote}
part a - Make the figure and get the values.
\end{quote}

Solution:

\begin{Shaded}
\begin{Highlighting}[]
\CommentTok{\# Scatterplots by dataset}
\FunctionTok{ggplot}\NormalTok{(datasaurus\_dozen, }\FunctionTok{aes}\NormalTok{(}\AttributeTok{x =}\NormalTok{ x, }\AttributeTok{y =}\NormalTok{ y, }\AttributeTok{color =}\NormalTok{ dataset))}\SpecialCharTok{+}
  \FunctionTok{geom\_point}\NormalTok{()}\SpecialCharTok{+}
  \FunctionTok{facet\_wrap}\NormalTok{(}\SpecialCharTok{\textasciitilde{}}\NormalTok{ dataset, }\AttributeTok{ncol =} \DecValTok{3}\NormalTok{) }\SpecialCharTok{+}
  \FunctionTok{theme}\NormalTok{(}\AttributeTok{legend.position =} \StringTok{"none"}\NormalTok{)}

\CommentTok{\# Correlations by dataset}
\NormalTok{datasaurus\_dozen }\SpecialCharTok{\%\textgreater{}\%}
  \FunctionTok{group\_by}\NormalTok{(dataset) }\SpecialCharTok{\%\textgreater{}\%}
  \FunctionTok{summarize}\NormalTok{(}\AttributeTok{r =} \FunctionTok{cor}\NormalTok{(x, y)) }\SpecialCharTok{\%\textgreater{}\%}
  \FunctionTok{kable}\NormalTok{(}\AttributeTok{booktabs =} \ConstantTok{TRUE}\NormalTok{, }\AttributeTok{digits =} \DecValTok{3}\NormalTok{)}
\end{Highlighting}
\end{Shaded}

\begin{quote}
part b - What do you notice? Is the correlation coefficient an
appropriate summary for any of these datasets? Why or why not?
\end{quote}

Solution:

\begin{quote}
part c - What do you think \texttt{ncol\ =\ 3} does in the code above?
What about \texttt{digits\ =\ 3}?
\end{quote}

Solution:

\newpage

\hypertarget{more-on-r}{%
\section{4 - More on R}\label{more-on-r}}

\begin{quote}
part a - Code chunk options
\end{quote}

.Qmds/.Rmds allow for global options to be set for how code chunks work.
At times, you might change those options for a particular chunk, such as
when you adjusted fig.width and fig.height above. In .Qmds, these end up
inside the chunk, while in .Rmds, these are in the \{r\} definition of
the chunk.

Let's explore how these options work when compared to the default
settings. Add the following options (one at a time!) to the code chunk
below and re-knit the PDF each time. Try to identify what each option is
doing to the PDF output. The QMD and RMD versions are provided for you
to see (note the differences in capitalization required), but this is a
.Qmd, so use those.

\begin{itemize}
\tightlist
\item
  \texttt{\#\textbar{}\ echo:\ false} OR Rmd version
  \texttt{echo\ =\ FALSE}:
\item
  \texttt{\#\textbar{}\ eval:\ false} OR Rmd version
  \texttt{eval\ =\ FALSE}:
\item
  \texttt{\#\textbar{}\ include:\ false} OR Rmd version
  \texttt{include\ =\ FALSE}:
\item
  \texttt{\#\textbar{}\ collapse:\ true} OR Rmd version
  \texttt{collapse\ =\ TRUE}:
\end{itemize}

After trying out the options, to complete this part, use appropriate
options to output the correlations but not show any of the code used to
get them.

Solution:

\begin{Shaded}
\begin{Highlighting}[]
\CommentTok{\# Correlations by dataset (plain R output)}
\NormalTok{datasaurus\_dozen }\SpecialCharTok{\%\textgreater{}\%}
  \FunctionTok{group\_by}\NormalTok{(dataset) }\SpecialCharTok{\%\textgreater{}\%}
  \FunctionTok{summarize}\NormalTok{(}\AttributeTok{r =} \FunctionTok{cor}\NormalTok{(x, y) }\SpecialCharTok{\%\textgreater{}\%} \FunctionTok{round}\NormalTok{(}\DecValTok{3}\NormalTok{)) }
\end{Highlighting}
\end{Shaded}

\begin{verbatim}
# A tibble: 13 x 2
   dataset         r
   <chr>       <dbl>
 1 away       -0.064
 2 bullseye   -0.069
 3 circle     -0.068
 4 dino       -0.064
 5 dots       -0.06 
 6 h_lines    -0.062
 7 high_lines -0.069
 8 slant_down -0.069
 9 slant_up   -0.069
10 star       -0.063
11 v_lines    -0.069
12 wide_lines -0.067
13 x_shape    -0.066
\end{verbatim}

Note: There are a lot more R code chunk options, most of which we'll not
use in this course. But if you're interested in the full list, check out
the
\href{https://www.rstudio.com/wp-content/uploads/2015/03/rmarkdown-reference.pdf}{RMarkdown
Reference Guide}.

\begin{quote}
part b - Inline R code
\end{quote}

Here's another fun functionality of RMarkdown: You can use inline R code
chunks to place R numerical output in your text. For instance:

The dino dataset contains 142 observations and has 3 variables. The mean
\(x\) value is 54.2632732 units and the mean \(y\) value is 47.8322528
units.

You can use the \texttt{round} command to get a sensible number of
digits: The mean \(x\) value is 54.3 units and the mean \(y\) value is
47.8 units.

Use inline R coding to write a sentence about the standard deviation of
\(x\) in the dino dataset reported to 1 decimal place.

Solution:

\begin{quote}
part c - Text versus code formatting
\end{quote}

Markdown will take text that is typed outside a code chunk and format it
so that it doesn't run off the page. It doesn't do that within code
chunks though. This can be problematic for both code and code comments.
We'll be learning a coding style that should help avoid issues with
this, but if you have a lot of options, it can still be an issue. Let's
look at some code that needs adjusting to fix.

(If you don't have a code margin line showing in your RStudio, look
under Tools \textgreater{} Global Options \textgreater{} Code
\textgreater{} Display for \texttt{show\ margin} and set the margin
column to 80. It's closer to 88, but if you use 80, it will match the
text margin.)

We will use the \texttt{iris} data set built into R for this example (it
has more variables to play with). Look at the help menu for details on
the data set.

If you knit to pdf without doing anything in the chunk below, both the
comment and code will run off the page. (Knit and check that you can see
this!) We want to fix this.

For comments, you can spread them on different lines manually by making
a series of shorter messages (demo-ed above in problem 2) or simply by
writing a more compact comment.

For code, the idea is to only allow for one pipe (the \%\textgreater\%
operator) or plus sign (for plots) per line, using indents to keep the
overall line together (see examples above). This is the coding style we
are using for class. It helps with code readability a ton! If you have a
long variable list or something (not pipes or plus signs causing the
issue), you'll need to manually hit enter somewhere in the list that's
not over the line.

Fix the code chunk below so that neither the comment (which you can
rewrite) nor the code goes off the edge of the page, following the
coding style described and demonstrated above.

Solution:

\begin{Shaded}
\begin{Highlighting}[]
\CommentTok{\# want to get the correlation between Sepal.Length and Sepal.Width for all three iris species in the data set}
\FunctionTok{data}\NormalTok{(iris)}
\NormalTok{iris }\SpecialCharTok{\%\textgreater{}\%} \FunctionTok{select}\NormalTok{(Species, Sepal.Length, Sepal.Width) }\SpecialCharTok{\%\textgreater{}\%} \FunctionTok{group\_by}\NormalTok{(Species) }\SpecialCharTok{\%\textgreater{}\%} \FunctionTok{summarize}\NormalTok{(}\AttributeTok{correlation =} \FunctionTok{cor}\NormalTok{(Sepal.Length, Sepal.Width)) }
\end{Highlighting}
\end{Shaded}

\begin{verbatim}
# A tibble: 3 x 2
  Species    correlation
  <fct>            <dbl>
1 setosa           0.743
2 versicolor       0.526
3 virginica        0.457
\end{verbatim}

\newpage

\textbf{References}

The original Datasaurus (\texttt{dino}) was created by Alberto Cairo in
\href{http://www.thefunctionalart.com/2016/08/download-datasaurus-never-trust-summary.html}{this
great blog post}. The other Dozen were generated using simulated
annealing and the process is described in the paper \emph{Same Stats,
Different Graphs: Generating Datasets with Varied Appearance and
Identical Statistics through Simulated Annealing} by Justin Matejka and
George Fitzmaurice. In the paper, the authors simulate a variety of
datasets that the same summary statistics to the Datasaurus but have
very different distributions.



\end{document}
